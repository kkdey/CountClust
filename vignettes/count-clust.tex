%\VignetteEngine{knitr::knitr}
%\VignetteIndexEntry{Grade of Membership Clustering and Visualization using CountClust}
%\VignettePackage{CountClust}

% To compile this document
% library('knitr'); rm(list=ls()); knit('CountClust/vignettes/count-clust.Rnw')
% library('knitr'); rm(list=ls()); knit2pdf('CountClust/vignettes/count-clust.Rnw'); openPDF('count-clust.pdf')
%

\documentclass[12pt]{article}

\newcommand{\CountClust}{\textit{CountClust}}
\usepackage{dsfont}
\usepackage{cite}




\RequirePackage{/Library/Frameworks/R.framework/Versions/3.2/Resources/library/BiocStyle/resources/tex/Bioconductor}

\AtBeginDocument{\bibliographystyle{/Library/Frameworks/R.framework/Versions/3.2/Resources/library/BiocStyle/resources/tex/unsrturl}}


\author{Kushal K Dey, Chiaowen Joyce Hsiao \& Matthew Stephens \\[1em] \small{\textit{Stephens Lab}, The University of Chicago} \mbox{ }\\ \small{\texttt{$^*$Correspondending Email: mstephens@uchicago.edu}}}

\bioctitle[Grade of Membership Clustering and Visualization using \CountClust{}]{Grade of Membership Model and Visualization for RNA-seq data using \CountClust{}}

\begin{document}

\maketitle

\begin{abstract}
  \vspace{1em}
 Grade of membership or GoM models (also known as admixture models or Latent Dirichlet Allocation") are a generalization of cluster models that allow each sample to have membership in multiple clusters. It is widely used to model ancestry of individuals in population genetics based on SNP/ microsatellite data and also in natural language processing for modeling documents \cite{Pritchard2000, Blei2003}.

This \R{} package implements tools to visualize the clusters obtained from fitting topic models using a Structure plot \cite{Rosenberg2002} and extract the top features/genes that distinguish the clusters. In presence of known technical or batch effects, the package also allows for correction of these confounding effects.

\vspace{1em}
\textbf{\CountClust{} version:} 0.1.0 \footnote{This document used the vignette from \Bioconductor{} package \Biocpkg{DESeq2, cellTree} as \CRANpkg{knitr} template}
\end{abstract}




\newpage

\tableofcontents

\section{Introduction}

In the context of RNA-seq expression (bulk or singlecell seq) data, the grade of membership model allows each sample (usually a tissue sample or a single cell) to have some proportion of its RNA-seq reads coming from each cluster. For typical bulk RNA-seq experiments this assumption
can be argued as follows: each tissue sample is a mixture of different cell types, and so clusters could represent cell types (which are determined by the expression patterns of the genes), and the membership of a sample in each cluster could represent the proportions of each cell type present in that sample.

Many software packages available for document clustering are applicable to modeling RNA-seq data. Here, we use the R package {\tt maptpx} \cite{Taddy2012} to fit these models, and we add functionality for visualizing the results and annotating clusters by their most distinctive genes to help biological interpretation. We also provide additional functionality to correct for batch effects and also compare the outputs from two different grade of membership model fits to the same set of samples but different in terms of feature description or model assumptions.

\section{\CountClust{} Installation}

\CountClust{} requires the following CRAN-R packages: \CRANpkg{maptpx}, \CRANpkg{slam},  \CRANpkg{ggplot2}, \CRANpkg{cowplot},\CRANpkg{parallel} along with the \Bioconductor{} package: \Biocpkg{limma}.

Installing \CountClust{} from \Bioconductor{} will install all these dependencies:

\begin{knitrout}
\definecolor{shadecolor}{rgb}{0.969, 0.969, 0.969}\color{fgcolor}\begin{kframe}
\begin{alltt}
\hlkwd{source}\hlstd{(}\hlstr{"http://bioconductor.org/biocLite.R"}\hlstd{)}
\hlkwd{biocLite}\hlstd{(}\hlstr{"CountClust"}\hlstd{)}
\end{alltt}
\end{kframe}
\end{knitrout}

For installing the working version of this package and loading the data required for this vignette, we use CRAN-R package \CRANpkg{devtools}.

\begin{knitrout}
\definecolor{shadecolor}{rgb}{0.969, 0.969, 0.969}\color{fgcolor}\begin{kframe}
\begin{alltt}
\hlkwd{library}\hlstd{(devtools)}
\hlkwd{install_github}\hlstd{(}\hlstr{'kkdey/CountClust'}\hlstd{)}
\end{alltt}
\end{kframe}
\end{knitrout}

Then load the package with:

\begin{knitrout}
\definecolor{shadecolor}{rgb}{0.969, 0.969, 0.969}\color{fgcolor}\begin{kframe}
\begin{alltt}
\hlkwd{library}\hlstd{(CountClust)}
\end{alltt}
\end{kframe}
\end{knitrout}

\section{Data Preparation}

We install the data packages as \begin{verb} expressionSet \end{verb} objects for bulk-RNA reads data from GTEx (Genotype Tissue Expression) V6 Project Brain tissue samples \cite{GTEX2013} and a singlecell-RNA reads data due to Deng \textit{et al} 2014 \cite{Deng2014}.


\begin{knitrout}
\definecolor{shadecolor}{rgb}{0.969, 0.969, 0.969}\color{fgcolor}\begin{kframe}
\begin{alltt}
\hlkwd{library}\hlstd{(devtools)}
\hlkwd{install_github}\hlstd{(}\hlstr{'kkdey/singleCellRNASeqMouseDeng2014'}\hlstd{)}
\hlkwd{install_github}\hlstd{(}\hlstr{'kkdey/GTExV6Brain'}\hlstd{)}
\end{alltt}
\end{kframe}
\end{knitrout}

\subsubsection{Deng et al 2014}

Load the scRNA-seq data due to Deng \textit{et al} 2014.

\begin{knitrout}
\definecolor{shadecolor}{rgb}{0.969, 0.969, 0.969}\color{fgcolor}\begin{kframe}
\begin{alltt}
\hlkwd{library}\hlstd{(singleCellRNASeqMouseDeng2014)}
\hlstd{deng.counts} \hlkwb{<-} \hlkwd{exprs}\hlstd{(Deng2014MouseESC)}
\hlstd{deng.meta_data} \hlkwb{<-} \hlkwd{pData}\hlstd{(Deng2014MouseESC)}
\hlstd{deng.gene_names} \hlkwb{<-} \hlkwd{rownames}\hlstd{(deng.counts)}
\end{alltt}
\end{kframe}
\end{knitrout}

\subsubsection{GTEx V6 Brain}

Load the bulk-RNA seq data from GTEx V6 brain data.

\begin{knitrout}
\definecolor{shadecolor}{rgb}{0.969, 0.969, 0.969}\color{fgcolor}\begin{kframe}
\begin{alltt}
\hlkwd{library}\hlstd{(GTExV6Brain)}
\hlstd{gtex.counts} \hlkwb{<-} \hlkwd{exprs}\hlstd{(GTExV6Brain)}
\hlstd{gtex.meta_data} \hlkwb{<-} \hlkwd{pData}\hlstd{(GTExV6Brain)}
\hlstd{gtex.gene_names} \hlkwb{<-} \hlkwd{rownames}\hlstd{(gtex.counts)}
\end{alltt}
\end{kframe}
\end{knitrout}


\section{Fitting the GoM Model}

We use a wrapper function of the \textit{topics()} function in the \CRANpkg{maptpx} due to Matt Taddy \cite{Taddy2012}.

As an example, we fit the topic model for \Robject{k}=4 on the GTEx V6 Brain data and save the GoM model output file to user-defined directory.

\begin{knitrout}
\definecolor{shadecolor}{rgb}{0.969, 0.969, 0.969}\color{fgcolor}\begin{kframe}
\begin{alltt}
\hlkwd{StructureObj}\hlstd{(}\hlkwd{t}\hlstd{(gtex.counts),}
            \hlkwc{nclus_vec}\hlstd{=}\hlnum{4}\hlstd{,} \hlkwc{tol}\hlstd{=}\hlnum{0.1}\hlstd{,}
             \hlkwc{path_rda}\hlstd{=}\hlstr{"../data/MouseDeng2014-topicFit.rda"}\hlstd{)}
\end{alltt}
\end{kframe}
\end{knitrout}

One can also input a vector of clusters under \begin{verb} nclus_vec \end{verb} as we do for a list of cluster numbers from $2$ to $7$ for Deng \textit{et al} 2014 data.

\begin{knitrout}
\definecolor{shadecolor}{rgb}{0.969, 0.969, 0.969}\color{fgcolor}\begin{kframe}
\begin{alltt}
\hlkwd{StructureObj}\hlstd{(}\hlkwd{t}\hlstd{(deng.counts),}
            \hlkwc{nclus_vec}\hlstd{=}\hlnum{2}\hlopt{:}\hlnum{7}\hlstd{,} \hlkwc{tol}\hlstd{=}\hlnum{0.1}\hlstd{,}
             \hlkwc{path_rda}\hlstd{=}\hlstr{"../data/MouseDeng2014-topicFit.rda"}\hlstd{)}
\end{alltt}
\end{kframe}
\end{knitrout}



\section{Structure plot visualization}

Now we perform the visualization for \Robject{k}=$6$ for the Deng \textit{et al} 2014 data.

\begin{knitrout}
\definecolor{shadecolor}{rgb}{0.969, 0.969, 0.969}\color{fgcolor}\begin{kframe}
\begin{alltt}
\hlstd{MouseDeng2014.topicFit} \hlkwb{<-} \hlkwd{get}\hlstd{(}\hlkwd{load}\hlstd{(}\hlstr{"../data/MouseDeng2014-topicFit.rda"}\hlstd{))}
\hlkwd{names}\hlstd{(MouseDeng2014.topicFit}\hlopt{$}\hlstd{clust_6)}
\end{alltt}
\begin{verbatim}
## [1] "K"     "theta" "omega" "BF"    "D"     "X"
\end{verbatim}
\begin{alltt}
\hlstd{omega} \hlkwb{<-} \hlstd{MouseDeng2014.topicFit}\hlopt{$}\hlstd{clust_6}\hlopt{$}\hlstd{omega}


\hlstd{annotation} \hlkwb{<-} \hlkwd{data.frame}\hlstd{(}
  \hlkwc{sample_id} \hlstd{=} \hlkwd{paste0}\hlstd{(}\hlstr{"X"}\hlstd{,} \hlkwd{c}\hlstd{(}\hlnum{1}\hlopt{:}\hlkwd{NROW}\hlstd{(omega))),}
  \hlkwc{tissue_label} \hlstd{=} \hlkwd{factor}\hlstd{(}\hlkwd{rownames}\hlstd{(omega),}
                        \hlkwc{levels} \hlstd{=} \hlkwd{rev}\hlstd{(} \hlkwd{c}\hlstd{(}\hlstr{"zy"}\hlstd{,} \hlstr{"early2cell"}\hlstd{,}
                                        \hlstr{"mid2cell"}\hlstd{,} \hlstr{"late2cell"}\hlstd{,}
                                        \hlstr{"4cell"}\hlstd{,} \hlstr{"8cell"}\hlstd{,} \hlstr{"16cell"}\hlstd{,}
                                        \hlstr{"earlyblast"}\hlstd{,}\hlstr{"midblast"}\hlstd{,}
                                         \hlstr{"lateblast"}\hlstd{) ) ) )}

\hlkwd{rownames}\hlstd{(omega)} \hlkwb{<-} \hlstd{annotation}\hlopt{$}\hlstd{sample_id;}
\end{alltt}
\end{kframe}
\end{knitrout}

\begin{figure}[h]
\begin{center}
































